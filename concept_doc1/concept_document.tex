\documentclass{article}
\usepackage{hyperref}
\hypersetup{
    colorlinks=true,
    linkcolor=blue,
    filecolor=magenta,      
    urlcolor=cyan,
    pdftitle={Overleaf Example},
    pdfpagemode=FullScreen,
    }

\urlstyle{same}
\title{The Locust}
\author{Mackenzie Norman}
\begin{document}
\maketitle
\textit{}
\section*{Prior Reading}
\begin{list} {}
\item\textit{The Day of The Locust}. 

\item\textit{\href{https://en.wikipedia.org/wiki/Locust}{Wikipedia on Locusts}}. 
\item\textit{Nervous Conditions}. 
\end{list}
\section*{Concept}
The locust is a story of three characters, Daniel Paul, Eula, and K. intertwined, experiencing the world.\footnote{told from a free indirect view. You the narrator are the locust. Agitating the characters in the resource starved world.} Begins with K. at home. K. is down to his last beer and the gas station is already closed. Intro to context switching through train of thought\footnote{From Sand Dancer, I don't like the `brooding system'. I'd like for mine to be way more train of thought. - some character and time switching will happen here. perhaps Faulkners original idea for benjys chapter would help clarify things.} Daniel Paul is neurotic and recently getting out of being institutionalized. Eula is plump and radiant. An Abel to K.s Cain. 








\end{document}