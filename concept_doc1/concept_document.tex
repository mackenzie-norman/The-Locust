\documentclass{article}
\usepackage{hyperref}
\hypersetup{
    colorlinks=true,
    linkcolor=blue,
    filecolor=magenta,      
    urlcolor=cyan,
    pdftitle={Overleaf Example},
    pdfpagemode=FullScreen,
    }

\urlstyle{same}
\usepackage{tikz}
\input{story.tikzstyles}

\title{The Locust}
\author{Mackenzie Norman}
\begin{document}
\maketitle
\textit{}
\section*{Prior Reading}
\begin{list} {}
\item \textit{Memoirs of my nervous illness}
\item\textit{The Day of The Locust}. 
\item\textit{\href{https://en.wikipedia.org/wiki/Locust}{Wikipedia on Locusts}}. 

\end{list}
\section*{Concept}
The locust is a story of three characters, Daniel Paul, Eula, and K. intertwined, experiencing the world. 
Daniel Paul is neurotic and recently getting out of being institutionalized, His thoughts are strewn with footnotes. Eula is plump and radiant. An Abel to K.s Cain. K. appears to be the main character although he has perhaps the least agency. There is tension between K. and Eula although what isn't really known yet.

The entire story takes place over one night. With some flashbacks to moments not in the night but very little.
\paragraph{}
I want to inspect themes of destruction (The Day of the locust is arguably negative towards this,  but I think a cloud of locusts would be beautiful) as well as scarcity and violence 
\pagebreak
\section*{Beats}
\subsection*{Intro}  
Intro to scene/character switching through train of thought\footnote{From Sand Dancer, I don't like the `brooding system'. I'd like for mine to be way more train of thought. - some character and time switching will happen here. perhaps Faulkners original idea for benjys chapter would help clarify things.}

K. is down to his last beer and the gas station is already closed. As he heads out for more he thinks, which allows you to go to Eula or Daniel. 
\begin{center}
\resizebox{\textwidth}{!}{
    \begin{tikzpicture}
	\begin{pgfonlayer}{nodelayer}
		\node [style=Plot Point] (7) at (-22.75, 8.25) {Ks Intro};
		\node [style=Plot Point] (8) at (-19, 5.5) {Entry To Thinking};
		\node [style=Plot Point] (9) at (-14.25, 11.5) {Eulas Intro};
		\node [style=Plot Point] (10) at (-14.75, 0) {Daniels Intro};
		\node [style=Plot Point] (11) at (-5.75, 11.5) {Thinking as Eula};
		\node [style=Plot Point] (12) at (-6.25, 0) {Thinking as Daniel};
		\node [style=Plot Point] (15) at (-4.75, 4.25) {};
	\end{pgfonlayer}
	\begin{pgfonlayer}{edgelayer}
		\draw (12) to (15);
		\draw [style=Directed] (7) to (8);
		\draw [style=Directed] (7) to (8);
		\draw [style=Directed] (8) to (10);
		\draw [style=Directed] (8) to (9);
		\draw [style=Directed] (12) to (15);
		\draw [style=Directed] (9) to (11);
		\draw [style=Directed] (10) to (12);
		\draw [style=Directed] (12) to (9);
		\draw [style=Directed] (11) to (10);
	\end{pgfonlayer}
\end{tikzpicture}

}   
\end{center}

\subsection*{Transit}
\textit{``Fixed personality; unchanging masks; character is carving. When Marcus Aurelius says `carve your mask' he means `develop you character.' Stereotypes.''}
\paragraph{}
K. rides on his bike while thinking of what his performance is.
Eula rides the bus.
Daniel is at home.
This is all thought mostly. Some game-ifying of transit might happen.

\subsection*{Dancing}

\textit{``Another aspect of the colonized affectivity can be seen when it is drained of energy by the energy of dance\dots The colonized's way of relaxing is precisely this muscular orgy during which the brutal aggressiveness and impulsive violence are channeled.''} 
\paragraph{}

The climax of our story and a part where peoples thoughts return too and also causes peoples mind to wander. In some cases K. isn't here and in some he is. Same with Daniel.

\subsection*{The streets}
\textit{``The colonist's sector is a sector built to last, all stone and steel. It's a sector of lights and paved roads, where the trash cans constantly overflow with strange and wonderful garbage, undreamed-of leftovers.''}
\paragraph{}
As K. and Eula leave the nightclub they meet Daniel. They walk together through the streets. More of a falling action than anything. Lots of chances to return to the dance, mentally or otherwise.

\subsection*{Ending(s)}

I'm hoping this comes to me. I know it has to be destructive and due to scarcity. Humans change in one way or another. Eula - starves? Daniel - sane?  













\end{document}